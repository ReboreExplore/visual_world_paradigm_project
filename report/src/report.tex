% Options for packages loaded elsewhere
\PassOptionsToPackage{unicode}{hyperref}
\PassOptionsToPackage{hyphens}{url}
\PassOptionsToPackage{dvipsnames,svgnames,x11names}{xcolor}
%
\documentclass[
  a4paper,
]{article}

\usepackage{amsmath,amssymb}
\usepackage{setspace}
\usepackage{iftex}
\ifPDFTeX
  \usepackage[T1]{fontenc}
  \usepackage[utf8]{inputenc}
  \usepackage{textcomp} % provide euro and other symbols
\else % if luatex or xetex
  \usepackage{unicode-math}
  \defaultfontfeatures{Scale=MatchLowercase}
  \defaultfontfeatures[\rmfamily]{Ligatures=TeX,Scale=1}
\fi
\usepackage[]{times}
\ifPDFTeX\else  
    % xetex/luatex font selection
\fi
% Use upquote if available, for straight quotes in verbatim environments
\IfFileExists{upquote.sty}{\usepackage{upquote}}{}
\IfFileExists{microtype.sty}{% use microtype if available
  \usepackage[]{microtype}
  \UseMicrotypeSet[protrusion]{basicmath} % disable protrusion for tt fonts
}{}
\makeatletter
\@ifundefined{KOMAClassName}{% if non-KOMA class
  \IfFileExists{parskip.sty}{%
    \usepackage{parskip}
  }{% else
    \setlength{\parindent}{0pt}
    \setlength{\parskip}{6pt plus 2pt minus 1pt}}
}{% if KOMA class
  \KOMAoptions{parskip=half}}
\makeatother
\usepackage{xcolor}
\usepackage[margin=1in]{geometry}
\setlength{\emergencystretch}{3em} % prevent overfull lines
\setcounter{secnumdepth}{5}
% Make \paragraph and \subparagraph free-standing
\ifx\paragraph\undefined\else
  \let\oldparagraph\paragraph
  \renewcommand{\paragraph}[1]{\oldparagraph{#1}\mbox{}}
\fi
\ifx\subparagraph\undefined\else
  \let\oldsubparagraph\subparagraph
  \renewcommand{\subparagraph}[1]{\oldsubparagraph{#1}\mbox{}}
\fi

\usepackage{color}
\usepackage{fancyvrb}
\newcommand{\VerbBar}{|}
\newcommand{\VERB}{\Verb[commandchars=\\\{\}]}
\DefineVerbatimEnvironment{Highlighting}{Verbatim}{commandchars=\\\{\}}
% Add ',fontsize=\small' for more characters per line
\newenvironment{Shaded}{}{}
\newcommand{\AlertTok}[1]{\textcolor[rgb]{1.00,0.33,0.33}{\textbf{#1}}}
\newcommand{\AnnotationTok}[1]{\textcolor[rgb]{0.42,0.45,0.49}{#1}}
\newcommand{\AttributeTok}[1]{\textcolor[rgb]{0.84,0.23,0.29}{#1}}
\newcommand{\BaseNTok}[1]{\textcolor[rgb]{0.00,0.36,0.77}{#1}}
\newcommand{\BuiltInTok}[1]{\textcolor[rgb]{0.84,0.23,0.29}{#1}}
\newcommand{\CharTok}[1]{\textcolor[rgb]{0.01,0.18,0.38}{#1}}
\newcommand{\CommentTok}[1]{\textcolor[rgb]{0.42,0.45,0.49}{#1}}
\newcommand{\CommentVarTok}[1]{\textcolor[rgb]{0.42,0.45,0.49}{#1}}
\newcommand{\ConstantTok}[1]{\textcolor[rgb]{0.00,0.36,0.77}{#1}}
\newcommand{\ControlFlowTok}[1]{\textcolor[rgb]{0.84,0.23,0.29}{#1}}
\newcommand{\DataTypeTok}[1]{\textcolor[rgb]{0.84,0.23,0.29}{#1}}
\newcommand{\DecValTok}[1]{\textcolor[rgb]{0.00,0.36,0.77}{#1}}
\newcommand{\DocumentationTok}[1]{\textcolor[rgb]{0.42,0.45,0.49}{#1}}
\newcommand{\ErrorTok}[1]{\textcolor[rgb]{1.00,0.33,0.33}{\underline{#1}}}
\newcommand{\ExtensionTok}[1]{\textcolor[rgb]{0.84,0.23,0.29}{\textbf{#1}}}
\newcommand{\FloatTok}[1]{\textcolor[rgb]{0.00,0.36,0.77}{#1}}
\newcommand{\FunctionTok}[1]{\textcolor[rgb]{0.44,0.26,0.76}{#1}}
\newcommand{\ImportTok}[1]{\textcolor[rgb]{0.01,0.18,0.38}{#1}}
\newcommand{\InformationTok}[1]{\textcolor[rgb]{0.42,0.45,0.49}{#1}}
\newcommand{\KeywordTok}[1]{\textcolor[rgb]{0.84,0.23,0.29}{#1}}
\newcommand{\NormalTok}[1]{\textcolor[rgb]{0.14,0.16,0.18}{#1}}
\newcommand{\OperatorTok}[1]{\textcolor[rgb]{0.14,0.16,0.18}{#1}}
\newcommand{\OtherTok}[1]{\textcolor[rgb]{0.44,0.26,0.76}{#1}}
\newcommand{\PreprocessorTok}[1]{\textcolor[rgb]{0.84,0.23,0.29}{#1}}
\newcommand{\RegionMarkerTok}[1]{\textcolor[rgb]{0.42,0.45,0.49}{#1}}
\newcommand{\SpecialCharTok}[1]{\textcolor[rgb]{0.00,0.36,0.77}{#1}}
\newcommand{\SpecialStringTok}[1]{\textcolor[rgb]{0.01,0.18,0.38}{#1}}
\newcommand{\StringTok}[1]{\textcolor[rgb]{0.01,0.18,0.38}{#1}}
\newcommand{\VariableTok}[1]{\textcolor[rgb]{0.89,0.38,0.04}{#1}}
\newcommand{\VerbatimStringTok}[1]{\textcolor[rgb]{0.01,0.18,0.38}{#1}}
\newcommand{\WarningTok}[1]{\textcolor[rgb]{1.00,0.33,0.33}{#1}}

\providecommand{\tightlist}{%
  \setlength{\itemsep}{0pt}\setlength{\parskip}{0pt}}\usepackage{longtable,booktabs,array}
\usepackage{calc} % for calculating minipage widths
% Correct order of tables after \paragraph or \subparagraph
\usepackage{etoolbox}
\makeatletter
\patchcmd\longtable{\par}{\if@noskipsec\mbox{}\fi\par}{}{}
\makeatother
% Allow footnotes in longtable head/foot
\IfFileExists{footnotehyper.sty}{\usepackage{footnotehyper}}{\usepackage{footnote}}
\makesavenoteenv{longtable}
\usepackage{graphicx}
\makeatletter
\def\maxwidth{\ifdim\Gin@nat@width>\linewidth\linewidth\else\Gin@nat@width\fi}
\def\maxheight{\ifdim\Gin@nat@height>\textheight\textheight\else\Gin@nat@height\fi}
\makeatother
% Scale images if necessary, so that they will not overflow the page
% margins by default, and it is still possible to overwrite the defaults
% using explicit options in \includegraphics[width, height, ...]{}
\setkeys{Gin}{width=\maxwidth,height=\maxheight,keepaspectratio}
% Set default figure placement to htbp
\makeatletter
\def\fps@figure{htbp}
\makeatother
\newlength{\cslhangindent}
\setlength{\cslhangindent}{1.5em}
\newlength{\csllabelwidth}
\setlength{\csllabelwidth}{3em}
\newlength{\cslentryspacingunit} % times entry-spacing
\setlength{\cslentryspacingunit}{\parskip}
\newenvironment{CSLReferences}[2] % #1 hanging-ident, #2 entry spacing
 {% don't indent paragraphs
  \setlength{\parindent}{0pt}
  % turn on hanging indent if param 1 is 1
  \ifodd #1
  \let\oldpar\par
  \def\par{\hangindent=\cslhangindent\oldpar}
  \fi
  % set entry spacing
  \setlength{\parskip}{#2\cslentryspacingunit}
 }%
 {}
\usepackage{calc}
\newcommand{\CSLBlock}[1]{#1\hfill\break}
\newcommand{\CSLLeftMargin}[1]{\parbox[t]{\csllabelwidth}{#1}}
\newcommand{\CSLRightInline}[1]{\parbox[t]{\linewidth - \csllabelwidth}{#1}\break}
\newcommand{\CSLIndent}[1]{\hspace{\cslhangindent}#1}

\makeatletter
\@ifpackageloaded{tcolorbox}{}{\usepackage[skins,breakable]{tcolorbox}}
\@ifpackageloaded{fontawesome5}{}{\usepackage{fontawesome5}}
\definecolor{quarto-callout-color}{HTML}{909090}
\definecolor{quarto-callout-note-color}{HTML}{0758E5}
\definecolor{quarto-callout-important-color}{HTML}{CC1914}
\definecolor{quarto-callout-warning-color}{HTML}{EB9113}
\definecolor{quarto-callout-tip-color}{HTML}{00A047}
\definecolor{quarto-callout-caution-color}{HTML}{FC5300}
\definecolor{quarto-callout-color-frame}{HTML}{acacac}
\definecolor{quarto-callout-note-color-frame}{HTML}{4582ec}
\definecolor{quarto-callout-important-color-frame}{HTML}{d9534f}
\definecolor{quarto-callout-warning-color-frame}{HTML}{f0ad4e}
\definecolor{quarto-callout-tip-color-frame}{HTML}{02b875}
\definecolor{quarto-callout-caution-color-frame}{HTML}{fd7e14}
\makeatother
\makeatletter
\makeatother
\makeatletter
\makeatother
\makeatletter
\@ifpackageloaded{caption}{}{\usepackage{caption}}
\AtBeginDocument{%
\ifdefined\contentsname
  \renewcommand*\contentsname{Table of contents}
\else
  \newcommand\contentsname{Table of contents}
\fi
\ifdefined\listfigurename
  \renewcommand*\listfigurename{List of Figures}
\else
  \newcommand\listfigurename{List of Figures}
\fi
\ifdefined\listtablename
  \renewcommand*\listtablename{List of Tables}
\else
  \newcommand\listtablename{List of Tables}
\fi
\ifdefined\figurename
  \renewcommand*\figurename{Figure}
\else
  \newcommand\figurename{Figure}
\fi
\ifdefined\tablename
  \renewcommand*\tablename{Table}
\else
  \newcommand\tablename{Table}
\fi
}
\@ifpackageloaded{float}{}{\usepackage{float}}
\floatstyle{ruled}
\@ifundefined{c@chapter}{\newfloat{codelisting}{h}{lop}}{\newfloat{codelisting}{h}{lop}[chapter]}
\floatname{codelisting}{Listing}
\newcommand*\listoflistings{\listof{codelisting}{List of Listings}}
\makeatother
\makeatletter
\@ifpackageloaded{caption}{}{\usepackage{caption}}
\@ifpackageloaded{subcaption}{}{\usepackage{subcaption}}
\makeatother
\makeatletter
\@ifpackageloaded{tcolorbox}{}{\usepackage[skins,breakable]{tcolorbox}}
\makeatother
\makeatletter
\@ifundefined{shadecolor}{\definecolor{shadecolor}{rgb}{.97, .97, .97}}
\makeatother
\makeatletter
\makeatother
\makeatletter
\makeatother
\ifLuaTeX
  \usepackage{selnolig}  % disable illegal ligatures
\fi
\IfFileExists{bookmark.sty}{\usepackage{bookmark}}{\usepackage{hyperref}}
\IfFileExists{xurl.sty}{\usepackage{xurl}}{} % add URL line breaks if available
\urlstyle{same} % disable monospaced font for URLs
\hypersetup{
  pdftitle={Visual World Paradigm},
  pdfauthor={Pritom Gogoi, Manpa Barman, Kapil Chander Mulchandani},
  colorlinks=true,
  linkcolor={blue},
  filecolor={Maroon},
  citecolor={Blue},
  urlcolor={Blue},
  pdfcreator={LaTeX via pandoc}}

\title{\textbf{Visual World Paradigm}}
\usepackage{etoolbox}
\makeatletter
\providecommand{\subtitle}[1]{% add subtitle to \maketitle
  \apptocmd{\@title}{\par {\large #1 \par}}{}{}
}
\makeatother
\subtitle{\emph{A classical visual world study showing how people
predict upcoming words with the help of Gazepoint eye tracker}}
\author{Pritom Gogoi, Manpa Barman, Kapil Chander Mulchandani}
<<<<<<< HEAD
\date{2023-08-21}
=======
\date{2023-08-20}
>>>>>>> e730125 (added 50% experiment section to report)

\begin{document}
\maketitle
\begin{abstract}
The study presented in this paper explores the dynamics of predictive
language processing through the visual world paradigm (VWP), a widely
employed method in cognitive psychology. The primary objective of the
research is to unwind how individuals anticipate or predict forthcoming
words during the unfolding of the spoken instructions, leveraging the
Gazepoint eye tracker for precise gaze pattern analysis. The
investigation delves into the impact of competitor words on gaze
patterns, to study the cognitive mechanisms underlying real-time
language comprehension. Our experiment uses a collection of competitor
words sharing phonetic or semantic similarities with the target, and
validates the hypothesis that the existence of such competitors leads to
an increased number of fixations on them, reflecting the participants'
evolving predictions of the upcoming word.
\end{abstract}
<<<<<<< HEAD
\ifdefined\Shaded\renewenvironment{Shaded}{\begin{tcolorbox}[frame hidden, interior hidden, boxrule=0pt, borderline west={3pt}{0pt}{shadecolor}, sharp corners, breakable, enhanced]}{\end{tcolorbox}}\fi
=======
\ifdefined\Shaded\renewenvironment{Shaded}{\begin{tcolorbox}[boxrule=0pt, interior hidden, sharp corners, borderline west={3pt}{0pt}{shadecolor}, enhanced, breakable, frame hidden]}{\end{tcolorbox}}\fi
>>>>>>> e730125 (added 50% experiment section to report)

\renewcommand*\contentsname{Table of contents}
{
\hypersetup{linkcolor=}
\setcounter{tocdepth}{3}
\tableofcontents
}
\setstretch{1.25}
<<<<<<< HEAD
\href{./report/docs/report.pdf}{Download the pdf version.}
=======
\href{report/docs/report.pdf}{Download the pdf version.}
>>>>>>> e730125 (added 50% experiment section to report)

\hypertarget{introduction}{%
\section{Introduction}\label{introduction}}

\hypertarget{visual-world-paradigm}{%
\subsection{Visual World Paradigm}\label{visual-world-paradigm}}

The visual world paradigm is an experimental framework that investigates
language processing by monitoring participants' eye movements while they
interact with visual stimuli. Introduced by psychologists Richard Cooper
and Thomas P. McDermott in the late 1990s, this paradigm have been
continuosly refined and expanded, adapting it to different research
questions and using advancements in eye-tracking technology to gain
deeper insights into real-time language comprehension and visual
attention processes. Through this framework the researchers try to
simulate the integration of spoken language and visual information as
they naturally occur in everyday situations so that we can draw
inferences on the attention focus on specific objects in their visual
display over time.

\hypertarget{objective-of-our-project}{%
\subsection{Objective of our project}\label{objective-of-our-project}}

We try to answer the research question:

\emph{Does the presence of similar-sounding words influence our tendency
to focus on those words apart from the target words as the word
unfolds?}

Our project is to study the nature of spoken word recognition as the
word unfolds. Here the key aspect of the visual world paradigm is that
participants' eye movements serve as an index of their ongoing language
processing and interpretation.

We aim to explore two fundamental conclusions concerning spoken word
recognition and the underlying models, building upon the established
research in this domain:

\begin{itemize}
\tightlist
\item
  Spoken word recognition is dynamic in nature which suggests that
  listeners continuously update and refine their interpretations as more
  information becomes available.
\item
  Spoken word recognition models make assumptions that multiple
  candidates compete for recognition during the unfolding of the spoken
  word.
\end{itemize}

Paul D.Allopenna, James S. Magnuson and Michael K. Tanenhaus in their
paper \emph{``Tracking the Time Course of Spoken Word Recognition Using
Eye Movements: Evidence for Continuous Mapping Models''} investigated a
similar structure of the experiment and found the following results:

\begin{figure}

{\centering \includegraphics[width=0.5\textwidth,height=\textheight]{img/ref_graph.png}

}

\caption{\label{fig-matrix}Probability of fixating on each item type
over time in the full competitor condition}

\end{figure}

In this figure we have the probability of fixation on four words: -
Referent (e.g beaker) : Target Word - Cohort (e.g beetle) : Similar
Sounding Word - Rhyme (e.g speaker) : Rhyming word - Unrelated (e.g
carriage) : Unrelated word to the rest (phonetically or semantically.)

In the beginning the participants hear {[}bi{]}, which could be the
beginning of \emph{beaker} but also could be the beginning of
\emph{beetle}. So during the first 400 ms the particpants start looking
at both of those words, more than they look at the others. After some
time as they hear the {[}k{]} i.e.~now they are hearing {[}bik{]}, thus
they discard their choice of \emph{beetle} and stop looking at it. But
by the time they've heard the whole word \emph{beaker}, they might
realize that \emph{beaker} rhymes almost exactly with \emph{speaker} and
get confused about if they heard \emph{speaker} at the very first place.
For the last word carriage the pronunciation is totally unrelated to the
target \emph{beaker}, so there is a very less probability of the
participant actually fixation at the unrelated word.

<<<<<<< HEAD
\hypertarget{preprocessing-and-analysis}{%
\section{Preprocessing and Analysis}\label{preprocessing-and-analysis}}

\hypertarget{extracting-relevant-data-from-the-log-files}{%
\subsection{Extracting relevant data from the log
files}\label{extracting-relevant-data-from-the-log-files}}

The logs present in the .tsv files are important for our analysis. Apart
from containing the data recordings from the experiments, they also
contain the information about the individual trials. In the text block
below, a logs for a sample trial are shown.

\begin{Shaded}
\begin{Highlighting}[]
\NormalTok{START\_EXP}
\NormalTok{START\_TRIAL: 0 T: SADDLE.PNG R: PICKLE.PNG B: PADLOCK.PNG L: CANDY.PNG}
\NormalTok{FIXATE\_CENTER\_AUDIO\_ONSET, COND: 12 TARGET: CANDY}
\NormalTok{CENTRE\_GAZE\_START}
\NormalTok{INSTRUCTION\_TO\_CLICK\_ONSET}
\NormalTok{LOG\_AUDIO\_TARGET\_START}
\NormalTok{LOG\_AUDIO\_TARGET\_END}
\NormalTok{CLICK\_RESPONSE\_END}
\NormalTok{FINAL\_FIXATION\_START, SELECTED: CANDY.PNG}
\NormalTok{FINAL\_FIXATION\_END}
\NormalTok{….}
\NormalTok{….}
\NormalTok{….}
\NormalTok{….}
\NormalTok{STOP\_EXP}
\end{Highlighting}
\end{Shaded}

=======
>>>>>>> e730125 (added 50% experiment section to report)
\begin{center}\rule{0.5\linewidth}{0.5pt}\end{center}

References: (Vitay, 2017)

See Figure~\ref{fig-matrix} and Section~\ref{sec-results}.

\[
    \tau \, \frac{dx_j(t)}{dt} + x_j(t)= \sum_i w^{in}_{ij} \, r^{in}_i(t) + g \, \sum_{i \neq j} w^{rec}_{ij} \, r_i(t)
\]

<<<<<<< HEAD
\begin{tcolorbox}[enhanced jigsaw, bottomrule=.15mm, opacitybacktitle=0.6, coltitle=black, bottomtitle=1mm, rightrule=.15mm, colback=white, opacityback=0, leftrule=.75mm, arc=.35mm, toptitle=1mm, left=2mm, colbacktitle=quarto-callout-note-color!10!white, colframe=quarto-callout-note-color-frame, breakable, titlerule=0mm, title=\textcolor{quarto-callout-note-color}{\faInfo}\hspace{0.5em}{Nota Bene}, toprule=.15mm]
=======
\begin{tcolorbox}[enhanced jigsaw, titlerule=0mm, opacityback=0, colbacktitle=quarto-callout-note-color!10!white, bottomtitle=1mm, breakable, colback=white, opacitybacktitle=0.6, colframe=quarto-callout-note-color-frame, left=2mm, bottomrule=.15mm, arc=.35mm, title=\textcolor{quarto-callout-note-color}{\faInfo}\hspace{0.5em}{Nota Bene}, toptitle=1mm, rightrule=.15mm, toprule=.15mm, leftrule=.75mm, coltitle=black]
>>>>>>> e730125 (added 50% experiment section to report)

Important information.

\end{tcolorbox}

\begin{Shaded}
\begin{Highlighting}[]
\ControlFlowTok{for}\NormalTok{ i }\KeywordTok{in} \BuiltInTok{range}\NormalTok{(}\DecValTok{10}\NormalTok{):}
    \BuiltInTok{print}\NormalTok{(i)}
\end{Highlighting}
\end{Shaded}

\hypertarget{second-subsection}{%
\subsection{Second subsection}\label{second-subsection}}

\url{https://www.youtube.com/embed/tPgf_btTFlc}

\hypertarget{material-and-methods}{%
\section{Material and methods}\label{material-and-methods}}

\hypertarget{sec-results}{%
\section{Results}\label{sec-results}}

\hypertarget{discussion}{%
\section{Discussion}\label{discussion}}

\hypertarget{references}{%
\section*{References}\label{references}}
\addcontentsline{toc}{section}{References}

\hypertarget{refs}{}
\begin{CSLReferences}{1}{0}
\leavevmode\vadjust pre{\hypertarget{ref-Vitay2017}{}}%
Vitay, J. (2017). On the role of dopamine in motivated behavior: A
neuro-computational approach. Available at:
\url{https://julien-vitay.net/publication/vitay2017/}.

\end{CSLReferences}



\end{document}
